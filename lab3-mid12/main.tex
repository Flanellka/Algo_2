\documentclass[12pt,a4paper]{article}

% Настройки для поддержки русского языка
\usepackage[utf8]{inputenc}
\usepackage[T2A]{fontenc}
\usepackage[russian]{babel}

% Математические пакеты
\usepackage{amsmath, amssymb}

% Настройка полей
\usepackage{geometry}
\geometry{left=3cm, right=1.5cm, top=2cm, bottom=2cm}

\begin{document}

% Титульный лист
\begin{titlepage}
    \begin{center}
        \vspace*{5cm}
        
        \Large \textbf{Предмет: Алгоритмы и структуры данных}
        
        \vspace{2cm}
        
        \huge \textbf{Амортизационный анализ операций в красно-черных деревьях (Метод потенциалов)}
        
        \vspace{3cm}
        
        \Large Выполнила:\\
        \textbf{Петренко Екатерина Алексеевна}
        
        \vfill
        
        
    \end{center}
\end{titlepage}

\section*{Введение}
Задача состоит в доказательстве того, что амортизационное время операций вставки и удаления в красно-черном дереве (КЧД) составляет $O(\log N)$. Для доказательства используется метод потенциалов.

В красно-черном дереве из $N$ узлов максимальная высота ограничена сверху как $2 \log(N+1)$. Это гарантирует, что обычный поиск узла всегда занимает время $O(\log N)$.

\section*{1. Метод потенциалов и функция потенциала}
В методе потенциалов мы сопоставляем структуре данных $D$ на шаге $i$ некоторую функцию потенциала $\Phi(D_i)$. Амортизационная стоимость $i$-й операции определяется формулой:
$$a_i = c_i + \Phi(D_i) - \Phi(D_{i-1})$$
где:
\begin{itemize}
    \item $c_i$ — реальная стоимость операции;
    \item $\Phi(D_i) - \Phi(D_{i-1})$ — разность потенциалов между текущим состоянием (ближайшим соседом) и предыдущим.
\end{itemize}

Так как операции балансировки дерева в основном связаны с нарушениями свойств красных узлов, логично привязать функцию потенциала к их количеству. Определим потенциал дерева $T$ как:
$$\Phi(T) = k \cdot R(T)$$
где $R(T)$ — количество красных узлов в дереве, а $k$ — константа, покрывающая стоимость одной перекраски (для простоты примем $k=2$).

\section*{2. Анализ операции вставки}
Реальная стоимость вставки $c_i$ состоит из:
\begin{enumerate}
    \item Поиска позиции для нового узла: $O(\log N)$.
    \item Добавления узла и возможной балансировки (перекраски и повороты).
\end{enumerate}

При добавлении нового узла он всегда окрашивается в красный цвет. Изменение потенциала при самом добавлении составляет:
$$\Delta\Phi = 2(R+1) - 2R = 2$$
Это константное увеличение потенциала.

Далее может потребоваться балансировка (если родитель тоже красный).
\begin{itemize}
    \item \textbf{Перекраска (Recoloring):} Мы меняем цвет родителя и «дяди» с красного на черный, а «дедушку» — с черного на красный. В результате количество красных узлов уменьшается на 1.
    $$\Delta\Phi = -2$$
    Это уменьшение потенциала компенсирует реальную стоимость перекраски $O(1)$.
    \item \textbf{Поворот (Rotation):} Выполняется не более 2-х поворотов (стоимость $O(1)$), после чего свойства КЧД восстанавливаются, и алгоритм завершается. Изменение потенциала при этом также ограничено константой.
\end{itemize}

Итоговая амортизационная стоимость вставки:
$$a_{insert} = O(\log N) + O(1) + \Delta\Phi = O(\log N)$$

\section*{3. Анализ операции удаления}
Реальная стоимость удаления также начинается с поиска узла, что занимает $O(\log N)$.

\begin{itemize}
    \item Если удаляется красный узел, черная высота не меняется, балансировка не нужна. Изменение потенциала отрицательное $\Delta\Phi < 0$.
    \item Если удаляется черный узел, возникает нарушение «двойной черноты». Балансировка может подниматься вверх по дереву. Перекраски могут переводить черные узлы в красные, увеличивая потенциал, однако количество таких шагов строго ограничено высотой дерева, то есть $O(\log N)$.
    \item Поворотов при удалении выполняется не более трех, что дает константное время $O(1)$.
\end{itemize}

В худшем случае реальные затраты на балансировку и изменение потенциала пропорциональны высоте дерева:
$$a_{delete} = O(\log N) + \Delta\Phi_{max} = O(\log N) + O(\log N)$$
$$a_{delete} = O(\log N)$$

\section*{Заключение}
С помощью метода потенциалов, выбрав функцию, зависящую от количества красных узлов, мы доказали, что даже с учетом каскадных перекрасок при балансировке, амортизационное время операций вставки и удаления в красно-черном дереве строго ограничено логарифмической функцией от числа элементов и составляет $O(\log N)$.

\end{document}